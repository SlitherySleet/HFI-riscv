%% bare_conf_compsoc.tex
%% V1.4b
%% 2015/08/26
%% by Michael Shell
%% See:
%% http://www.michaelshell.org/
%% for current contact information.
%%
%% This is a skeleton file demonstrating the use of IEEEtran.cls
%% (requires IEEEtran.cls version 1.8b or later) with an IEEE Computer
%% Society conference paper.
%%
%% Support sites:
%% http://www.michaelshell.org/tex/ieeetran/
%% http://www.ctan.org/pkg/ieeetran
%% and
%% http://www.ieee.org/

%%*************************************************************************
%% Legal Notice:
%% This code is offered as-is without any warranty either expressed or
%% implied; without even the implied warranty of MERCHANTABILITY or
%% FITNESS FOR A PARTICULAR PURPOSE! 
%% User assumes all risk.
%% In no event shall the IEEE or any contributor to this code be liable for
%% any damages or losses, including, but not limited to, incidental,
%% consequential, or any other damages, resulting from the use or misuse
%% of any information contained here.
%%
%% All comments are the opinions of their respective authors and are not
%% necessarily endorsed by the IEEE.
%%
%% This work is distributed under the LaTeX Project Public License (LPPL)
%% ( http://www.latex-project.org/ ) version 1.3, and may be freely used,
%% distributed and modified. A copy of the LPPL, version 1.3, is included
%% in the base LaTeX documentation of all distributions of LaTeX released
%% 2003/12/01 or later.
%% Retain all contribution notices and credits.
%% ** Modified files should be clearly indicated as such, including  **
%% ** renaming them and changing author support contact information. **
%%*************************************************************************


% *** Authors should verify (and, if needed, correct) their LaTeX system  ***
% *** with the testflow diagnostic prior to trusting their LaTeX platform ***
% *** with production work. The IEEE's font choices and paper sizes can   ***
% *** trigger bugs that do not appear when using other class files.       ***                          ***
% The testflow support page is at:
% http://www.michaelshell.org/tex/testflow/



\documentclass[conference,compsoc]{IEEEtran}
% Some/most Computer Society conferences require the compsoc mode option,
% but others may want the standard conference format.
%
% If IEEEtran.cls has not been installed into the LaTeX system files,
% manually specify the path to it like:
% \documentclass[conference,compsoc]{../sty/IEEEtran}





% Some very useful LaTeX packages include:
% (uncomment the ones you want to load)


% *** MISC UTILITY PACKAGES ***
%
%\usepackage{ifpdf}
% Heiko Oberdiek's ifpdf.sty is very useful if you need conditional
% compilation based on whether the output is pdf or dvi.
% usage:
% \ifpdf
%   % pdf code
% \else
%   % dvi code
% \fi
% The latest version of ifpdf.sty can be obtained from:
% http://www.ctan.org/pkg/ifpdf
% Also, note that IEEEtran.cls V1.7 and later provides a builtin
% \ifCLASSINFOpdf conditional that works the same way.
% When switching from latex to pdflatex and vice-versa, the compiler may
% have to be run twice to clear warning/error messages.






% *** CITATION PACKAGES ***
%
\ifCLASSOPTIONcompsoc
  % IEEE Computer Society needs nocompress option
  % requires cite.sty v4.0 or later (November 2003)
  \usepackage[nocompress]{cite}
\else
  % normal IEEE
  \usepackage{cite}
\fi
% cite.sty was written by Donald Arseneau
% V1.6 and later of IEEEtran pre-defines the format of the cite.sty package
% \cite{} output to follow that of the IEEE. Loading the cite package will
% result in citation numbers being automatically sorted and properly
% "compressed/ranged". e.g., [1], [9], [2], [7], [5], [6] without using
% cite.sty will become [1], [2], [5]--[7], [9] using cite.sty. cite.sty's
% \cite will automatically add leading space, if needed. Use cite.sty's
% noadjust option (cite.sty V3.8 and later) if you want to turn this off
% such as if a citation ever needs to be enclosed in parenthesis.
% cite.sty is already installed on most LaTeX systems. Be sure and use
% version 5.0 (2009-03-20) and later if using hyperref.sty.
% The latest version can be obtained at:
% http://www.ctan.org/pkg/cite
% The documentation is contained in the cite.sty file itself.
%
% Note that some packages require special options to format as the Computer
% Society requires. In particular, Computer Society  papers do not use
% compressed citation ranges as is done in typical IEEE papers
% (e.g., [1]-[4]). Instead, they list every citation separately in order
% (e.g., [1], [2], [3], [4]). To get the latter we need to load the cite
% package with the nocompress option which is supported by cite.sty v4.0
% and later.





% *** GRAPHICS RELATED PACKAGES ***
%
\ifCLASSINFOpdf
  % \usepackage[pdftex]{graphicx}
  % declare the path(s) where your graphic files are
  % \graphicspath{{../pdf/}{../jpeg/}}
  % and their extensions so you won't have to specify these with
  % every instance of \includegraphics
  % \DeclareGraphicsExtensions{.pdf,.jpeg,.png}
\else
  % or other class option (dvipsone, dvipdf, if not using dvips). graphicx
  % will default to the driver specified in the system graphics.cfg if no
  % driver is specified.
  % \usepackage[dvips]{graphicx}
  % declare the path(s) where your graphic files are
  % \graphicspath{{../eps/}}
  % and their extensions so you won't have to specify these with
  % every instance of \includegraphics
  % \DeclareGraphicsExtensions{.eps}
\fi
% graphicx was written by David Carlisle and Sebastian Rahtz. It is
% required if you want graphics, photos, etc. graphicx.sty is already
% installed on most LaTeX systems. The latest version and documentation
% can be obtained at: 
% http://www.ctan.org/pkg/graphicx
% Another good source of documentation is "Using Imported Graphics in
% LaTeX2e" by Keith Reckdahl which can be found at:
% http://www.ctan.org/pkg/epslatex
%
% latex, and pdflatex in dvi mode, support graphics in encapsulated
% postscript (.eps) format. pdflatex in pdf mode supports graphics
% in .pdf, .jpeg, .png and .mps (metapost) formats. Users should ensure
% that all non-photo figures use a vector format (.eps, .pdf, .mps) and
% not a bitmapped formats (.jpeg, .png). The IEEE frowns on bitmapped formats
% which can result in "jaggedy"/blurry rendering of lines and letters as
% well as large increases in file sizes.
%
% You can find documentation about the pdfTeX application at:
% http://www.tug.org/applications/pdftex





% *** MATH PACKAGES ***
%
%\usepackage{amsmath}
% A popular package from the American Mathematical Society that provides
% many useful and powerful commands for dealing with mathematics.
%
% Note that the amsmath package sets \interdisplaylinepenalty to 10000
% thus preventing page breaks from occurring within multiline equations. Use:
%\interdisplaylinepenalty=2500
% after loading amsmath to restore such page breaks as IEEEtran.cls normally
% does. amsmath.sty is already installed on most LaTeX systems. The latest
% version and documentation can be obtained at:
% http://www.ctan.org/pkg/amsmath





% *** SPECIALIZED LIST PACKAGES ***
%
%\usepackage{algorithmic}
% algorithmic.sty was written by Peter Williams and Rogerio Brito.
% This package provides an algorithmic environment fo describing algorithms.
% You can use the algorithmic environment in-text or within a figure
% environment to provide for a floating algorithm. Do NOT use the algorithm
% floating environment provided by algorithm.sty (by the same authors) or
% algorithm2e.sty (by Christophe Fiorio) as the IEEE does not use dedicated
% algorithm float types and packages that provide these will not provide
% correct IEEE style captions. The latest version and documentation of
% algorithmic.sty can be obtained at:
% http://www.ctan.org/pkg/algorithms
% Also of interest may be the (relatively newer and more customizable)
% algorithmicx.sty package by Szasz Janos:
% http://www.ctan.org/pkg/algorithmicx




% *** ALIGNMENT PACKAGES ***
%
%\usepackage{array}
% Frank Mittelbach's and David Carlisle's array.sty patches and improves
% the standard LaTeX2e array and tabular environments to provide better
% appearance and additional user controls. As the default LaTeX2e table
% generation code is lacking to the point of almost being broken with
% respect to the quality of the end results, all users are strongly
% advised to use an enhanced (at the very least that provided by array.sty)
% set of table tools. array.sty is already installed on most systems. The
% latest version and documentation can be obtained at:
% http://www.ctan.org/pkg/array


% IEEEtran contains the IEEEeqnarray family of commands that can be used to
% generate multiline equations as well as matrices, tables, etc., of high
% quality.




% *** SUBFIGURE PACKAGES ***
%\ifCLASSOPTIONcompsoc
%  \usepackage[caption=false,font=footnotesize,labelfont=sf,textfont=sf]{subfig}
%\else
%  \usepackage[caption=false,font=footnotesize]{subfig}
%\fi
% subfig.sty, written by Steven Douglas Cochran, is the modern replacement
% for subfigure.sty, the latter of which is no longer maintained and is
% incompatible with some LaTeX packages including fixltx2e. However,
% subfig.sty requires and automatically loads Axel Sommerfeldt's caption.sty
% which will override IEEEtran.cls' handling of captions and this will result
% in non-IEEE style figure/table captions. To prevent this problem, be sure
% and invoke subfig.sty's "caption=false" package option (available since
% subfig.sty version 1.3, 2005/06/28) as this is will preserve IEEEtran.cls
% handling of captions.
% Note that the Computer Society format requires a sans serif font rather
% than the serif font used in traditional IEEE formatting and thus the need
% to invoke different subfig.sty package options depending on whether
% compsoc mode has been enabled.
%
% The latest version and documentation of subfig.sty can be obtained at:
% http://www.ctan.org/pkg/subfig




% *** FLOAT PACKAGES ***
%
%\usepackage{fixltx2e}
% fixltx2e, the successor to the earlier fix2col.sty, was written by
% Frank Mittelbach and David Carlisle. This package corrects a few problems
% in the LaTeX2e kernel, the most notable of which is that in current
% LaTeX2e releases, the ordering of single and double column floats is not
% guaranteed to be preserved. Thus, an unpatched LaTeX2e can allow a
% single column figure to be placed prior to an earlier double column
% figure.
% Be aware that LaTeX2e kernels dated 2015 and later have fixltx2e.sty's
% corrections already built into the system in which case a warning will
% be issued if an attempt is made to load fixltx2e.sty as it is no longer
% needed.
% The latest version and documentation can be found at:
% http://www.ctan.org/pkg/fixltx2e


%\usepackage{stfloats}
% stfloats.sty was written by Sigitas Tolusis. This package gives LaTeX2e
% the ability to do double column floats at the bottom of the page as well
% as the top. (e.g., "\begin{figure*}[!b]" is not normally possible in
% LaTeX2e). It also provides a command:
%\fnbelowfloat
% to enable the placement of footnotes below bottom floats (the standard
% LaTeX2e kernel puts them above bottom floats). This is an invasive package
% which rewrites many portions of the LaTeX2e float routines. It may not work
% with other packages that modify the LaTeX2e float routines. The latest
% version and documentation can be obtained at:
% http://www.ctan.org/pkg/stfloats
% Do not use the stfloats baselinefloat ability as the IEEE does not allow
% \baselineskip to stretch. Authors submitting work to the IEEE should note
% that the IEEE rarely uses double column equations and that authors should try
% to avoid such use. Do not be tempted to use the cuted.sty or midfloat.sty
% packages (also by Sigitas Tolusis) as the IEEE does not format its papers in
% such ways.
% Do not attempt to use stfloats with fixltx2e as they are incompatible.
% Instead, use Morten Hogholm'a dblfloatfix which combines the features
% of both fixltx2e and stfloats:
%
% \usepackage{dblfloatfix}
% The latest version can be found at:
% http://www.ctan.org/pkg/dblfloatfix




% *** PDF, URL AND HYPERLINK PACKAGES ***
%
%\usepackage{url}
% url.sty was written by Donald Arseneau. It provides better support for
% handling and breaking URLs. url.sty is already installed on most LaTeX
% systems. The latest version and documentation can be obtained at:
% http://www.ctan.org/pkg/url
% Basically, \url{my_url_here}.




% *** Do not adjust lengths that control margins, column widths, etc. ***
% *** Do not use packages that alter fonts (such as pslatex).         ***
% There should be no need to do such things with IEEEtran.cls V1.6 and later.
% (Unless specifically asked to do so by the journal or conference you plan
% to submit to, of course. )


% correct bad hyphenation here
\hyphenation{op-tical net-works semi-conduc-tor}


\begin{document}
%
% paper title
% Titles are generally capitalized except for words such as a, an, and, as,
% at, but, by, for, in, nor, of, on, or, the, to and up, which are usually
% not capitalized unless they are the first or last word of the title.
% Linebreaks \\ can be used within to get better formatting as desired.
% Do not put math or special symbols in the title.
\title{HFI in RISC-V}


% author names and affiliations
% use a multiple column layout for up to three different
% affiliations
\author{\IEEEauthorblockN{Dinesh Keserla (dk27833)}
\IEEEauthorblockA{University of Texas at Austin\\
dineshkeserla@utexas.edu}
\and
\IEEEauthorblockN{Dev Patel (dbp748)}
\IEEEauthorblockA{University of Texas at Austin\\
devpatel199@utexas.edu}
\and
\IEEEauthorblockN{Joseph Stanley (jps4455)}
\IEEEauthorblockA{University of Texas at Austin\\
stanley.p.joseph@gmail.com}}

% conference papers do not typically use \thanks and this command
% is locked out in conference mode. If really needed, such as for
% the acknowledgment of grants, issue a \IEEEoverridecommandlockouts
% after \documentclass

% for over three affiliations, or if they all won't fit within the width
% of the page (and note that there is less available width in this regard for
% compsoc conferences compared to traditional conferences), use this
% alternative format:
% 
%\author{\IEEEauthorblockN{Michael Shell\IEEEauthorrefmark{1},
%Homer Simpson\IEEEauthorrefmark{2},
%James Kirk\IEEEauthorrefmark{3}, 
%Montgomery Scott\IEEEauthorrefmark{3} and
%Eldon Tyrell\IEEEauthorrefmark{4}}
%\IEEEauthorblockA{\IEEEauthorrefmark{1}School of Electrical and Computer Engineering\\
%Georgia Institute of Technology,
%Atlanta, Georgia 30332--0250\\ Email: see http://www.michaelshell.org/contact.html}
%\IEEEauthorblockA{\IEEEauthorrefmark{2}Twentieth Century Fox, Springfield, USA\\
%Email: homer@thesimpsons.com}
%\IEEEauthorblockA{\IEEEauthorrefmark{3}Starfleet Academy, San Francisco, California 96678-2391\\
%Telephone: (800) 555--1212, Fax: (888) 555--1212}
%\IEEEauthorblockA{\IEEEauthorrefmark{4}Tyrell Inc., 123 Replicant Street, Los Angeles, California 90210--4321}}




% use for special paper notices
%\IEEEspecialpapernotice{(Invited Paper)}




% make the title area
\maketitle

% As a general rule, do not put math, special symbols or citations
% in the abstract
% \begin{abstract}
% TODO
% \end{abstract}

% no keywords




% For peer review papers, you can put extra information on the cover
% page as needed:
% \ifCLASSOPTIONpeerreview
% \begin{center} \bfseries EDICS Category: 3-BBND \end{center}
% \fi
%
% For peerreview papers, this IEEEtran command inserts a page break and
% creates the second title. It will be ignored for other modes.
\IEEEpeerreviewmaketitle


\section{Project Goals}
Hardware-assisted fault isolation (HFI) on RISC-V aims to address two critical needs in modern computing systems: preventing security bugs by isolating untrusted or risky code, and better understanding (and improving) performance overheads that arise from current software-based or OS-based sandboxing mechanisms. By offering a minimal set of architectural extensions for in-process isolation, HFI empowers developers to sandbox components (e.g., libraries, plugins, user-defined functions) without resorting to heavyweight processes or virtual machines. Ultimately, the project's goal is to demonstrate a secure yet performant solution that seamlessly integrates into existing RISC-V designs.

Several recent high-profile vulnerabilities—such as Heartbleed (CVE-2014-0160) and Spectre (CVE-2017-5753, CVE-2017-5715)—underscore the need for stronger isolation between mutually distrusting code modules. System architects and software developers alike have turned to hardware features for efficient compartmentalization. In this spirit, our HFI design for RISC-V builds on prior work that introduced HFI for x86-64 architectures, enabling fast transitions into and out of "sandbox" contexts while enforcing strong data and control-flow checks.

\section{Background}
\subsection{Software-based Fault Isolation}
Modern applications consistently rely on a large number of various libraries for key functionality, however, this leads to a problem as bugs within these libraries can become a vector of attack for the entire application.
To mitigate this, it is critical to contain the scope of what each library can access or affect.
Traditionally, general isolation was considered at the granularity of processes, containers, and virtual machines, which do achieve the functionality of isolation of libraries through hardware-level protection via the MMU and user-kernel privileges.
Although such these isolation methods are functional, they have significant performance overhead from the context switching and memory management. 
Due to the fact that the executing libraries are quite small, it motivates isolation that is fine-grained and lightweight.

These issues are remedied by Software-based Fault Isolation (SFI), which performs isolation in userspace via compiler instrumentation.
SFI creates a sandbox, which is a restricted execution environment where untrusted code is contained in regards to both memory safety and control-flow integrity.
Memory accesses are restricted via bounds checks or address masking, while control-flow integrity is maintained by constraining jumps to valid, pre-approved destinations.
Additionally, to prevent unauthorized interaction with the operating system, system calls from sandboxed code are redirected to a trusted runtime, which will validate and mediate them before execution.

This process of fine-grained isolation within a single address space of the application is referred as \textit{in-process isolation.}
Recent SFI approaches, including WebAssembly's use of guard regions and JIT-based SFI schemes, have shown that in-process isolation is viable but remains limited by performance bottlenecks and compatibility issues. 
For instance, large FaaS (Function-as-a-Service) platforms and browser engines isolate tens of thousands of small code modules, resulting in high context-switch overheads or inordinate virtual-address consumption.
More explicitly, SFI systems face the following limitations:
\begin{enumerate}
  \item Performance overhead: SFI implementations can impose significant runtime overhead with over $40\%$ in practical workloads. This stems from the cost of inserting and executing additional instructions for validating memory loads and stores, control-flow validation, and system call interposition. \\
  \item Limited scalability: SFI schemes rely on allocating large, sparsely-mapped virtual memory regions with guard pages (8 GiB per sandbox in WebAssembly) in order to validate memory with the much more performant masking as opposed to base and bound checks. This consumes vast amounts of virtual address space, making it impractical to host many concurrent sandboxes. \\
  \item Poor compatibility: SFI requires code to be compiled or transformed in specific ways, which limits its compatibility with unmodified native binaries, low-level system libraries, and code that performs dynamic memory management or emits code at runtime. \\
  \item Vulnerability to spectre attacks: SFI checks are instrumented as checks within the code and like any other code, modern processors may speculatively execute instructions past these checks, potentially leaking sensitive data through a side channel. 
\end{enumerate}

Despite these limitations, SFI is increasingly adopted due to its relatively low overhead compared to traditional isolation mechanisms.

\subsection{HFI: Hardware-based Fault Isolation}
To address the aforementioned limitations found in SFI, Hardware-assisted Fault Isolation (HFI)\cite{HFI} was introduced. By consolidating and standardizing architectural support for these needs, HFI reduces reliance on kernel calls, shrinks the required memory footprint, and helps unify existing SFI-based runtimes under a lightweight, hardware-managed sandboxing mechanism.

Although hardware-based isolation solutions can be integrated at multiple levels, the approach mentioned in \cite{HFI} focuses on tightening the security of low-level software runtimes responsible for loading, executing, and managing untrusted or semi-trusted code within the same process. 
Hardware-assisted Fault Isolation (HFI) relies on the following core mechanisms:
\begin{enumerate}
    \item In-Process Isolation via Regions: The processor tracks a small set of metadata registers—each defining accessible addresses and permissions for code or data. When enabled, any out-of-bounds access or forbidden system call triggers an immediate trap, ensuring robust isolation. These hardware-enforced bounds checks happen in parallel with memory lookups. \\
    \item Lightweight Sandbox Mode: A single hfi\_enter instruction enables isolation, while hfi\_exit disables it. Each sandbox can optionally redirect system calls, enabling complete interposition if the sandbox is untrusted. The overhead of enter/exit can be on the order of tens of cycles, much lower than classical kernel-based context switches. \\
    \item User-Space Sandbox Configuration: By placing HFI instructions at the user level, ring transitions for memory setup are avoided. This design helps ephemeral tasks or microservices that create and destroy sandboxes frequently, as reconfiguring regions or resizing memory is a matter of writing to hardware registers, not calling into the OS. \\
    \item Spectre Mitigations: HFI includes optional serialization on transition instructions, ensuring that speculative execution cannot bypass newly set bounds or leak privileged data outside the sandbox. Additional hardware checks prevent malicious code from polluting caches or branch predictors with secrets from other regions. \\
\end{enumerate}

As demonstrated in prior x86 implementations \cite{HFI}, HFI's hardware cost is kept minimal, and the main processor pipelines remain largely untouched. This ensures that non-sandboxed code experiences negligible slowdowns, while sandboxed code reaps a significant performance advantage compared to purely software-based solutions.

\subsection{QEMU}

Quick Emulator (QEMU) is a free and open-source emulator for computer processors of varying architecture. 
QEMU is able to emulate various popular architectures such as x86, ARM, RISC-V, and more.
The core functionality of emulating various CPU architectures is achieved through \textit{dynamic binary translation}, in which guest machine instructions are converted to equivalent host machine instructions at runtime. 
This allows it to execute guest programs portably on host systems with different instruction set architectures.

QEMU leverages a Just-In-Time (JIT) compiler called Tiny Code Generator (TCG), which translates guest instructions into host instructions.
This translation layer is extensible and can be used to add functionality to existing instructions or create new instructions.
Additionally, the state inside the emulated architecture can be modified to add or remove registers and prepare for functional semantics of potential hardware modifications.
Furthermore, since these modifications can be applied entirely in software, the need for RTL simulation or FPGA synthesis can be avoided, which would be significantly more complex and time-consuming.

While QEMU enables validation of functional correctness, it cannot capture microarchitectural vulnerabilities such as Spectre or Meltdown, which depend on speculative execution, out-of-order pipelines, and cache-based side channels. 
For this reason, QEMU is not suitable for evaluating timing attacks or microarchitectural side effects, but remains a powerful tool for verifying correctness at the architectural level.
Moreover, verfiying performance is not possible as QEMU lacks features for modeling cycles.

\section{Security/Performance Currently}
Software-only isolation approaches face significant security challenges. Even robust SFI frameworks, such as those used in browser environments, face challenges from side-channel attacks, especially under cross-sandbox conditions where memory layout or branch predictors leak information.

In contrast, a hardware-assisted approach can unify bounds checking and control-flow constraints at the processor level, closing off out-of-bounds reads or writes before they ever reach caches or translation lookaside buffers. Furthermore, hardware-based mediation of system calls—critical for sandboxing unmodified native libraries—promises to eliminate entire classes of bypass vulnerabilities.

Beyond security, performance remains a central issue. Current software-based guard-page or compiler-instrumented approaches can incur 20-40\% runtime overhead on CPU-intensive code, and ephemeral workloads (common in serverless contexts) may pay a high cost to instantiate and tear down per-sandbox memory. Typical solutions (e.g., mprotect, madvise) require system calls and TLB shootdowns that do not scale gracefully.

On RISC-V, these overheads become particularly relevant for microcontroller-class devices—where every cycle matters—and for large-scale data centers wanting to isolate thousands of workloads concurrently. As prior experiments with x86-based HFI show, carefully designed hardware instructions can reduce context-switch time to near that of a function call, while supporting user-space mechanisms for setting up sandbox memory regions. The challenge is thus to replicate and refine these ideas in a clean-slate RISC-V setting without compromising security or drastically altering the instruction set.

\section{Approach}
Our work brings HFI concepts to the RISC-V ecosystem by adapting the existing HFI design while fitting into RISC-V's existing privileged architecture. RISC-V presents both unique opportunities and challenges for HFI implementation. Its extensible ISA and clean-slate design allow for elegant integration of new instructions, while its simpler microarchitecture requires careful consideration of implementation details.

By leveraging RISC-V's custom opcode space, we can integrate the necessary HFI instructions without disrupting compatibility with existing software. Our implementation focuses on providing the essential HFI functionality while maintaining the performance benefits that make hardware-assisted isolation valuable.

Our approach aims to preserve the security guarantees demanded by modern sandboxing frameworks, while also reducing overheads that historically have hindered fine-grained in-process isolation in the RISC-V ecosystem.

Therefore, the RISC-V implementation in QEMU can be modified to add necessary HFI registers and instructions, allowing for validating HFI's ISA semantics and control-flow traps. 
Moreover, by implementing in QEMU, RISC-V binaries using the created HFI instructions can be verified for functional correctness.
This allows the validation of memory safety enforcement, instruction semantics, and sandbox behavior across a variety of test workloads.

% TODO add the opcode limitation for explicit regions and possible solutions

\section{Implementation \& Evaluation}
\subsection{Implementation - Intial Setup}
Since we were implenting our RISC-V HFI extension in QEMU, we cloned the current open-source QEMU repository and began work there. We began by taking time to understand how QEMU works, the purpose of the various repositories, and gaining a better understanding of RISC-V. We worked with the assumption that the system being run on always supports HFI but it is an easy addition to add any necessary hardware checks to validate that assumption.\\
Our work then began by creating a basic internal register called \textit{hfi\_status}, which serves as a flag bit for verifying whether HFI can be toggled on or off via our newly implemented instructions. Through the use of QEMU logging we display the value of \textit{hfi\_status} and use it as a validator that the instructions have run.
For our first instruction, we have \textit{hfi\_enter}, which sets \textit{hfi\_status} $=1$ when executed. 
Additionally, we have \textit{hfi\_exit}, which sets \textit{hfi\_status} $=0$.
RISC-V contains unmapped opcodes that can be used for the additional of new instructions.
In order to implement our instruction we have opted to use the free opcode $0001011$.
We are then able to differentiate between the two functions by modifying the funct3 value of the instructions such that \textit{hfi\_enter} has funct3 $= 000$ and \textit{hfi\_enter} has funct3 $= 001$. Adding the control and status register (CSR) \textit{hfi\_status} and implementing the \textit{hfi\_enter} and \textit{hfi\_exit} gave us basic functionality to enable and disable HFI.\\

\subsection{Implementation - Implicit Regions}
After testing whether we can toggle the \textit{hfi\_status} register properly, we moved on to implement implicit regions. As aforementioned, implicit regions are used for native, unmodified binaries and an important feature of HFI. The original HFI paper had four implicit data regions and two implicit code regions. In our implementation, we supported an arbitrary number of data and code regions which were defined by a global variable that could be adjusted as needed. We also tested using varying numbers of each region and ensured the HFI logic worked seamlessly with the defined number of regions. Implicit regions are defined by a base and a mask.\\
To store metadata about these regions, we created new structs \textit{HFIImplicitDataRegion} and \textit{HFIImplicitCodeRegion} which we stored in the CPU architecture state in arrays of length based on the number of regions we were supporting. To set these regions, we implemented the instruction \textit{hfi\_set\_region\_size} which took in a region number, base address, and mask and configured a new data or code region accordingly. Then, to set permissions for this region, we added a new instruction \textit{hfi\_set\_region\_permissions} which took a region number and a 64 bit integer encoding the permissions for that region. Since a single call to \textit{hfi\_set\_region\_permissions} will be used for a singular region (differentiated using the code region number), the permission encoding for each of the 3 types of region can also differ. Implicit data regions store the enabled bit at bit 7, read bit at bit 6, and write bit at bit 5. Implicit code regions store the enabled bit at bit 7 and executable bit at bit 6.\\
Although implicit regions are used for unmodified native binaries, we are still working with the assumption that the software setting up HFI for the binary is a trustable user who will setup the regions properly. However, one check we do insert is for uninitialized regions. Originally, if a region wasn't setup properly where \textit{hfi\_set\_region\_size} wasn't called but \textit{hfi\_set\_region\_permissions} was called, a region would still be configured with $0x0$ for the base and mask. This was an issue as such a region allowed any code/data addresses to pass through. Hence, to prevent this, we set the mask on implicit regions to $0x11111111$ initially which would prevent regions not set up properly to fail HFI checks.\\
To implement our instruction we have opted to use the free opcode $0001011$.
We are then able to differentiate between these functions and the two aforementioned functions (\textit{hfi\_enter} and \textit{hfi\_exit}) by modifying the funct3 value of the instructions such that \textit{hfi\_set\_region\_size} has funct3 $= 010$ and \textit{hfi\_set\_region\_permissions} has funct3 $= 011$.

\subsection{Implementation - Explicit Regions}
Next, we moved on to implement explicit regions by defining a new struct \textit{HFIExpllicitDataRegion} to store metadata about explicit regions. These structs were also stored in a new array in the CPU Architecture State of length based on the global variable indicating the number of regions we wanted to support. While the original HFI paper had the h-prefix \textit{mov} instruct (\textit{hmov}), we had h-prefix load and store instruction. This difference is due to the ISA differences between x86 and RISC-V. Hence, for our implementation, we focused on providing support for a single region and implemented the following instructions: \textit{hlb0}, \textit{hlh0}, \textit{hlw0}, \textit{hlbu0}, \textit{hlhu0}, \textit{hlwu0}, \textit{hld0}, \textit{hsb0}, \textit{hsh0}, \textit{hsw0}, and \textit{hsd0}.\\
For each of the load and store instructions, we would check the whether both the beginning of the access address and end of the address access fall within a given data region (region 0 for the h-prefix and 0-postfix instructions). If HFI is enabled and the load or store fall within a properly configured region, then the load or store will be allowed through and proceed to load or store the given address by passing on the access addresses to the corresponding non-HFI load or store instruction. If the check fails, HFI would trap and prevent access to the address.\\
Explicit data regions are configured using a base and a bound. To set these regions, we use the same \textit{hfi\_set\_region\_base} and \textit{hfi\_set\_region\_permissions} that we used for implicit regions. The value passed for a mask in implicit regions was instead used as the bound value when an explicit region was configured. For the permissions, explicit data regions store the enabled bit at bit 7, read bit at bit 6, write bit at bit 5, and large region bit at bit 4.\\
We tested the implementation using this single explicit data region. One obvious limitation is how we can scale this to work for multiple data regions, which can be a challenge due to the opcode limitation discussed in section 4. However, based on the feasibility of our implementation for the one data region and overall structure of the implementation, we believe our proposed ideas from section 4 can allow easy extension to support more explicit data regions.\\
For these new load instructions, we used the free opcode $0001011$. For the new store instructions, we used the free opcode $1011011$. You can find the funct3 values we used to differentiate between them in Table~\ref{tab:hfi_instructions}.

\begin{table}[]
\caption{Encodings of HFI instruction implemented}
\label{tab:hfi_instructions}
\begin{tabular}{|l|l|l|}
\hline
                                         &  opcode   & funct3   \\ \hline
\textit{hfi\_enter}                      &  0001011  &  000     \\ \hline
\textit{hfi\_exit}                       &  0001011  &  001     \\ \hline
\textit{hfi\_set\_region\_size}          &  0001011  &  010     \\ \hline
\textit{hfi\_set\_region\_permissions}   &  0001011  &  011     \\ \hline
\textit{hfi\_hlb0}                       &  0101011  &  000     \\ \hline
\textit{hfi\_hlh0}                       &  0101011  &  001     \\ \hline
\textit{hfi\_hlw0}                       &  0101011  &  010     \\ \hline
\textit{hfi\_hlbu0}                      &  0101011  &  100     \\ \hline
\textit{hfi\_hlhu0}                      &  0101011  &  101     \\ \hline
\textit{hfi\_hlwu0}                      &  0101011  &  110     \\ \hline
\textit{hfi\_hld0}                       &  0101011  &  011     \\ \hline
\textit{hfi\_hsb0}                       &  1011011  &  010     \\ \hline
\textit{hfi\_hsh0}                       &  1011011  &  011     \\ \hline
\textit{hfi\_hsw0}                       &  1011011  &  100     \\ \hline
\textit{hfi\_hsd0}                       &  1011011  &  000     \\ \hline
\end{tabular}
\end{table}

\subsection{Evaluation - Testing}
To validate our HFI implementation and ensure correct functionality across all features, we developed a comprehensive testing infrastructure. Our testing framework consists of a suite of assembly programs written specifically to test different aspects of HFI functionality on RISC-V. These tests are compiled using the RISC-V GCC toolchain and executed on our modified QEMU binary.\\
We use a modular testing approach where each assembly program focuses on testing a specific component or feature of HFI. Our test suite covers several key areas of functionality: basic HFI operations that verify the core instructions function correctly and properly toggle the HFI status; region configuration tests that ensure memory regions can be properly defined with appropriate size and permissions; memory access tests that verify operations within configured regions succeed while operations outside them fail appropriately; and explicit region access tests that exercise the h-prefix load/store instructions to validate boundary checking and permission enforcement. We also implemented control tests to establish baseline functionality in non-HFI contexts.\\
These test programs are written in RISC-V assembly and leverage a custom set of macros defined in our includes directory that encode our HFI instructions using the appropriate opcodes and operands. For example, our HFI enter macro encodes the instruction with opcode 0x0B and function code 0 along with the specified register operands. This approach allows us to write readable assembly that directly tests our custom instructions.\\
We developed a Makefile-based build system that compiles these assembly programs using the RISC-V toolchain (riscv64-unknown-elf-as and riscv64-unknown-elf-ld). The resulting ELF binaries are then executed on our custom-built QEMU with additional debug flags to capture instruction execution and CPU state information.\\
The test execution outputs are captured in log files, allowing us to analyze instruction execution, register state changes, and any traps or exceptions that occur during testing. This approach enables us to verify several critical aspects of the implementation: the HFI instructions execute with the expected opcodes, region configurations are properly stored in CPU state, memory accesses are correctly validated against region bounds, and violations trigger appropriate traps when they should occur.\\
Through this testing infrastructure, we confirmed the correctness of our implementation across various scenarios including successful sandboxing, proper isolation of memory regions, and correct handling of boundary conditions. The modular nature of our test suite allows us to incrementally validate features as they are implemented and regression-test them as the implementation evolves.

\subsection{Evaluation - wasm2c}
We wanted to additionally add some of the evaluation done by the original HFI paper. We attempted to modify wasm2c by building on the work done already. The wasm2c code converts wasm files to C source and header and we wanted to modify it to use our modified h-prefix load and store instructions. We were unable to complete this step unfortunately due to some issues with building and running the module. However, this doesn't detract from the implementation of HFI itself. Although it would have been valuable to have used our HFI implementation in wasm2c, our unit tests still comprehensively demonstrate that HFI is feasible and functional in RISC-V. Our implementation is able to achieve the sandboxing functionality we wanted and effectively shows that HFI can extend to other architectures beyond x86. Hence, our work sufficiently shows the feasibility of HFI in RISC-V.

\section{Future Work}

\subsection{Utilize the RISC-V HFI in SFI}
As previously mentioned, a natural next step is to integrate our RISC-V HFI instructions into an existing SFI compiler to validate the correctness of end-to-end sandboxing behavior. 
This would involve instrumenting memory accesses and control flow in compiler-generated code and replacing them with HFI-enforced operations. 
Such an integration would demonstrate the correctness of SFI compilers when relying on hardware enforced isolation mechanisms.

% TODO FIX this maybe if u explain earlier
Additionally, supporting integration of HFI instructions into existing SFI compilers requires certain specification-focused instructions that we did not implement, such as:

\begin{itemize}
  \item \texttt{hfi\_get\_version}
  \item \texttt{hfi\_get\_linear\_code\_range\_count}
  \item \texttt{hfi\_get\_linear\_data\_range\_count}
  \item \texttt{hfi\_get\_exit\_reason}
  \item \texttt{hfi\_get\_exit\_location}
  \item \textit{and more...}
\end{itemize}

These additional instructions can be added with minimal modification to the \texttt{funct7} field.

\subsection{Cycle Accurate Simulator}
A more comprehensive implementation for evaluation of security and performance in addition to ISA semantics can be considered in a cycle accurate simulator, such as gem5, as done in the original HFI paper \cite{HFI}.
Since we used QEMU, an environment where the serialization of \textit{hfi\_enter} and \textit{hfi\_exit} would not produce observable effects, we chose not to implement serialization for Spectre mitigation.
In contrast, a cycle-accurate simulator enables observation of microarchitectural behavior, allowing for both functional correctness validation and evaluation of speculative execution resistance.
Moreover, with accurate cycle simulator, the performance overhead for HFI implementation can be validated when utilized within a SFI system.
As an implementation in a cycle accurate simulator is comparatively much slower in code modification and execution as compared to QEMU, as a feasibility metric for correctness, an implementation in QEMU was ideal with respect to our circumstances.

\section{Related Work}
Our work on HFI for RSIC-V builds on prior work that established HFI for x86-64 architectures \cite{HFI}. HFI provides a simple ISA extension to support secure, flexible, and efficient in-process isolation, addressing limitations of existing software-based techniques, including runtime overheads, scalability limits, Spectre vulnerability, and compatibility issues.\\
There are also several alternate approaches have been developed over the years for fine-grained isolation involving hardware-assisted techniques. Some systems have proposed extensions to the page table metadata to store a per-sandbox ID checked by hardware, such as Donky \cite{Donky} or IMIX \cite{IMIX}. Donky is a hardware-software co-design providing strong in-process isolation based on dynamic memory protection domains using hardware-backed protection keys. Its implementation on RISC-V uses the reserved top 10 bits of page-table entries for protection keys, supporting 1024 keys per process. Donky distinguishes itself by managing protection key policies entirely in userspace via a dedicated control and status register (DKRU). Access to the DKRU is protected by a hardware call gate, which leverages the RISC-V N extension. This architecture provides substantially stronger security guarantees than systems like Intel MPK, where the policy register is unprivileged, allowing Donky to shield against arbitrary code execution without needing binary inspection or rewriting. Donky facilitates fast domain switches entirely in userspace, with minimal hardware additions on RISC-V. Donky has been implemented for both x86 (using MPK emulation) and RISC-V architectures, while IMIX has only been implemented on x86.\\
IMIX (In-Process Memory Isolation Extension) introduces a minimal x86 ISA extension for isolating sensitive data within a process \cite{IMIX}. Unlike many isolation mechanisms tied to a specific defense (e.g., MPK for protection keys or CET for shadow stacks), IMIX provides a mitigation-agnostic hardware primitive for protecting security-critical data. IMIX enables memory pages to be marked as isolated using a new permission flag. These pages can only be accessed with a special instruction, smov, introduced by IMIX. This mechanism allows a wide range of memory-corruption defenses (such as CPI or CFI) to protect their internal metadata from arbitrary access by potentially compromised application code. IMIX was prototyped using Intel’s simulation infrastructure, with extensions to LLVM and Linux to support its new instruction semantics. The authors demonstrated that IMIX significantly reduces the attack surface for in-process defenses with low overhead and better flexibility than existing solutions like MPK, which are limited by unprivileged key registers and require complex mitigation workarounds to be secure.\\
In addition to this, there are also methods that make use of capability-based addressing, a scheme for controlling access to memory, which is seen most prominently in CHERI \cite{CHERI}. While initially developed as a hybrid capability model extending the 64-bit MIPS ISA and demonstrated on FPGA, the CHERI approach is described as not being specific to MIPS and is a prominent capability system relevant to discussions of advanced ISAs, including RISC-V (as referenced in related work comparisons). CHERI provides byte-granularity memory protection and enables hardware enforcement of memory safety and fault isolation. It utilizes a capability coprocessor and tagged memory, where capabilities are stored in dedicated registers and memory locations, protected by hardware tags and unforgeable manipulation instructions. Capabilities are unforgeable and define protection domains based on reachability. CHERI aims for a high level of source-code and binary compatibility relative to prior pure capability machines, supporting incremental deployment. However, adapting and recompiling application code is generally required to utilize its full protection features, and achieving comprehensive object-level memory safety involves extensive modifications throughout the software stack (OS, ABI, compiler). CHERI offers strong fine-grained protection within an address space and supports scalable userspace protection domains.\\
In summary, while Donky leverages protection keys and a userspace monitor with hardware protection for the policy register, and CHERI introduces capabilities for fine-grained, unforgeable memory references requiring significant software stack changes, IMIX proposes a lightweight ISA extension with minimal hardware cost to securely restrict access to sensitive memory used by diverse in-process defenses. Our work on HFI for RISC-V focuses on a region-based mechanism, minimal hardware modifications, userspace-only operation, and explicit Spectre mitigation for flexible and efficient in-process isolation supporting both adapted and unmodified code.


% conference papers do not normally have an appendix



% use section* for acknowledgment
\ifCLASSOPTIONcompsoc
  % The Computer Society usually uses the plural form
  \section*{Acknowledgments}
\else
  % regular IEEE prefers the singular form
  \section*{Acknowledgment}
\fi





% trigger a \newpage just before the given reference
% number - used to balance the columns on the last page
% adjust value as needed - may need to be readjusted if
% the document is modified later
%\IEEEtriggeratref{8}
% The "triggered" command can be changed if desired:
%\IEEEtriggercmd{\enlargethispage{-5in}}

% references section

% can use a bibliography generated by BibTeX as a .bbl file
% BibTeX documentation can be easily obtained at:
% http://mirror.ctan.org/biblio/bibtex/contrib/doc/
% The IEEEtran BibTeX style support page is at:
% http://www.michaelshell.org/tex/ieeetran/bibtex/
%\bibliographystyle{IEEEtran}
% argument is your BibTeX string definitions and bibliography database(s)
%\bibliography{IEEEabrv,../bib/paper}
%
% <OR> manually copy in the resultant .bbl file
% set second argument of \begin to the number of references
% (used to reserve space for the reference number labels box)
\begin{thebibliography}{1}

\bibitem{HFI}
Shravan Narayan, Tal Garfinkel, Mohammadkazem Taram, Joey Rudek, Daniel Moghimi, Evan Johnson, Chris Fallin, Anjo Vahldiek-Oberwagner, Michael LeMay, Ravi Sahita, Dean Tullsen, and Deian Stefan. 2023. Going beyond the Limits of SFI: Flexible and Secure Hardware-Assisted In-Process Isolation with HFI. In Proceedings of the 28th ACM International Conference on Architectural Support for Programming Languages and Operating Systems, Volume 3 (ASPLOS 2023). Association for Computing Machinery, New York, NY, USA, 266-281. https://doi.org/10.1145/3582016.3582023

\bibitem{Donky}
David Schrammel, Samuel Weiser, Stefan Steinegger, Martin Schwarzl, Michael
Schwarz, Stefan Mangard, and Daniel Gruss. 2020. Donky: Domain Keys-Efficient
In-Process Isolation for RISC-V and x86. In Proceedings of the USENIX Security
Symposium (USENIX Security).

\bibitem{CHERI}
Jonathan Woodruff, Robert N.M. Watson, David Chisnall, Simon W. Moore,
Jonathan Anderson, Brooks Davis, Ben Laurie, Peter G. Neumann, Robert Norton, and Michael Roe. 2014. The CHERI Capability Model: Revisiting RISC
in an Age of Risk. SIGARCH Comput. Archit. News 42, 3 (2014). https:
//doi.org/10.1145/2678373.2665740

\bibitem{IMIX}
Tommaso Frassetto, Patrick Jauernig, Christopher Liebchen, and Ahmad-Reza
Sadeghi. 2018. IMIX:In-Process Memory Isolation EXtension. In Proceedings of
the USENIX Security Symposium (USENIX Security).

\bibitem{QEMU}
About qemu (no date) About QEMU - QEMU documentation. Available at: https://www.qemu.org/docs/master/about/index.html (Accessed: 26 March 2025). 

\bibitem{ResearchGate}
Performance comparison between models and other simulators We evaluated the performance. Retrieved from https://www.researchgate.net/figure/Performance-comparison-between-models-and-other-simulators-We-evaluated-the-performance-fig2-341640101

\bibitem{Opcode}
Anon. Adding a custom instruction in the RVI subset. Retrieved March 26, 2025 from https://pcotret.gitlab.io/riscv-custom/adding-custom.htmlcustom-opcodes 

\end{thebibliography}




% that's all folks
\end{document}




% An example of a floating figure using the graphicx package.
% Note that \label must occur AFTER (or within) \caption.
% For figures, \caption should occur after the \includegraphics.
% Note that IEEEtran v1.7 and later has special internal code that
% is designed to preserve the operation of \label within \caption
% even when the captionsoff option is in effect. However, because
% of issues like this, it may be the safest practice to put all your
% \label just after \caption rather than within \caption{}.
%
% Reminder: the "draftcls" or "draftclsnofoot", not "draft", class
% option should be used if it is desired that the figures are to be
% displayed while in draft mode.
%
%\begin{figure}[!t]
%\centering
%\includegraphics[width=2.5in]{myfigure}
% where an .eps filename suffix will be assumed under latex, 
% and a .pdf suffix will be assumed for pdflatex; or what has been declared
% via \DeclareGraphicsExtensions.
%\caption{Simulation results for the network.}
%\label{fig_sim}
%\end{figure}

% Note that the IEEE typically puts floats only at the top, even when this
% results in a large percentage of a column being occupied by floats.


% An example of a double column floating figure using two subfigures.
% (The subfig.sty package must be loaded for this to work.)
% The subfigure \label commands are set within each subfloat command,
% and the \label for the overall figure must come after \caption.
% \hfil is used as a separator to get equal spacing.
% Watch out that the combined width of all the subfigures on a 
% line do not exceed the text width or a line break will occur.
%
%\begin{figure*}[!t]
%\centering
%\subfloat[Case I]{\includegraphics[width=2.5in]{box}%
%\label{fig_first_case}}
%\hfil
%\subfloat[Case II]{\includegraphics[width=2.5in]{box}%
%\label{fig_second_case}}
%\caption{Simulation results for the network.}
%\label{fig_sim}
%\end{figure*}
%
% Note that often IEEE papers with subfigures do not employ subfigure
% captions (using the optional argument to \subfloat[]), but instead will
% reference/describe all of them (a), (b), etc., within the main caption.
% Be aware that for subfig.sty to generate the (a), (b), etc., subfigure
% labels, the optional argument to \subfloat must be present. If a
% subcaption is not desired, just leave its contents blank,
% e.g., \subfloat[].


% An example of a floating table. Note that, for IEEE style tables, the
% \caption command should come BEFORE the table and, given that table
% captions serve much like titles, are usually capitalized except for words
% such as a, an, and, as, at, but, by, for, in, nor, of, on, or, the, to
% and up, which are usually not capitalized unless they are the first or
% last word of the caption. Table text will default to \footnotesize as
% the IEEE normally uses this smaller font for tables.
% The \label must come after \caption as always.
%
%\begin{table}[!t]
%% increase table row spacing, adjust to taste
%\renewcommand{\arraystretch}{1.3}
% if using array.sty, it might be a good idea to tweak the value of
% \extrarowheight as needed to properly center the text within the cells
%\caption{An Example of a Table}
%\label{table_example}
%\centering
%% Some packages, such as MDW tools, offer better commands for making tables
%% than the plain LaTeX2e tabular which is used here.
%\begin{tabular}{|c||c|}
%\hline
%One & Two\\
%\hline
%Three & Four\\
%\hline
%\end{tabular}
%\end{table}


% Note that the IEEE does not put floats in the very first column
% - or typically anywhere on the first page for that matter. Also,
% in-text middle ("here") positioning is typically not used, but it
% is allowed and encouraged for Computer Society conferences (but
% not Computer Society journals). Most IEEE journals/conferences use
% top floats exclusively. 
% Note that, LaTeX2e, unlike IEEE journals/conferences, places
% footnotes above bottom floats. This can be corrected via the
% \fnbelowfloat command of the stfloats package.
